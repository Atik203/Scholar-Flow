\chapter{Complete Feature List}
\label{ch:feature-list}

% NOTE: This chapter contains detailed feature documentation
% It should be expanded with content from PROJECT_REPORT.md Section 7
% For brevity in this demo, creating structured outline with first few features

\section{Overview}
\label{sec:features-overview}

\projectname{} \version{} implements 15 major feature categories, each with comprehensive functionality designed to address specific research workflow needs. This chapter provides detailed documentation of each feature with user workflows, technical implementation details, and usage examples.

% ============================================
% FEATURE 1: AUTHENTICATION
% ============================================
\section{User Authentication System}
\label{sec:feature-auth}

\subsection{Overview}

Multi-provider authentication system supporting email/password and OAuth (Google, GitHub).

\subsection{Key Capabilities}

\begin{itemize}[leftmargin=*]
    \item Email/password registration with validation
    \item Google OAuth 2.0 integration
    \item GitHub OAuth integration
    \item JWT-based session management (RS256)
    \item Email verification system
    \item Password reset with secure tokens
    \item Account linking for OAuth providers
\end{itemize}

\subsection{User Workflow}

\begin{enumerate}[leftmargin=*]
    \item \textbf{Registration:} User navigates to \code{/register}
    \item \textbf{Validation:} Zod schema validates email, password strength
    \item \textbf{Email Verification:} System sends verification email
    \item \textbf{Account Activation:} User clicks link to activate
    \item \textbf{Login:} JWT token issued (7-day expiry)
    \item \textbf{Refresh:} Automatic token refresh before expiry
\end{enumerate}

\subsection{Technical Implementation}

\begin{lstlisting}[language=JavaScript, caption={Authentication Middleware}]
// JWT verification with NextAuth
export const auth = async (req, res, next) => {
  const token = req.headers.authorization?.split(' ')[1];
  if (!token) throw new ApiError(401, 'Unauthorized');
  
  const decoded = await verifyJWT(token);
  const user = await prisma.$queryRaw`
    SELECT * FROM "User" WHERE id = ${decoded.userId}`;
  
  if (!user) throw new ApiError(401, 'Invalid token');
  req.user = user;
  next();
};
\end{lstlisting}

\subsection{Screenshots}

\begin{figure}[H]
\centering
\texttt{[PLACEHOLDER - Screenshot: Register Page]}
\caption{User Registration Form}
\label{fig:auth-register}
\end{figure}

% ============================================
% FEATURE 2: PAPER MANAGEMENT
% ============================================
\section{Paper Upload \& Management}
\label{sec:feature-paper-management}

\subsection{Overview}

Comprehensive paper lifecycle management from upload to deletion.

\subsection{Key Capabilities}

\begin{itemize}[leftmargin=*]
    \item PDF and DOCX upload (max 25MB)
    \item Automatic metadata extraction (title, authors, year)
    \item DOCX to PDF conversion via Gotenberg
    \item AWS S3 storage with CloudFront CDN
    \item Paper tagging and categorization
    \item Full-text search across papers
    \item Bulk operations (upload, tag, delete)
\end{itemize}

\subsection{Upload Workflow}

\begin{enumerate}[leftmargin=*]
    \item User selects file(s) from file picker
    \item Frontend validates file size and type
    \item File uploaded to backend with metadata
    \item Backend generates S3 presigned URL
    \item File uploaded directly to S3
    \item If DOCX: Gotenberg converts to PDF
    \item Metadata extracted and stored in PostgreSQL
    \item Paper added to user's default workspace
\end{enumerate}

\subsection{Search \& Filtering}

\begin{table}[H]
\centering
\caption{Paper Search Filters}
\label{tab:paper-filters}
\begin{tabular}{@{}lp{8cm}@{}}
\toprule
\textbf{Filter} & \textbf{Description} \\
\midrule
Query & Full-text search in title, authors, abstract \\
Tags & Filter by user-assigned tags \\
Date Range & Papers published between dates \\
Workspace & Papers in specific workspace \\
Collection & Papers in specific collection \\
Uploader & Papers uploaded by specific user \\
\bottomrule
\end{tabular}
\end{table}

% ============================================
% ADDITIONAL FEATURES (Continued...)
% ============================================

\section{Advanced Search \& Filtering}
\label{sec:feature-search}

% Content from PROJECT_REPORT.md Section 7.3

\section{Collection Management}
\label{sec:feature-collections}

% Content from PROJECT_REPORT.md Section 7.4

\section{Rich Text Editor}
\label{sec:feature-editor}

% Content from PROJECT_REPORT.md Section 7.5

\section{AI-Powered Features}
\label{sec:feature-ai}

% Content from PROJECT_REPORT.md Section 7.6

\section{Workspace Collaboration}
\label{sec:feature-workspace}

% Content from PROJECT_REPORT.md Section 7.7

\section{Subscription \& Billing}
\label{sec:feature-billing}

% Content from PROJECT_REPORT.md Section 7.8

\section{Admin Dashboard \& Analytics}
\label{sec:feature-admin}

% Content from PROJECT_REPORT.md Section 7.9

\section{Annotation \& Research Notes}
\label{sec:feature-annotations}

% Content from PROJECT_REPORT.md Section 7.10

\section{Citation Export}
\label{sec:feature-citations}

% Content from PROJECT_REPORT.md Section 7.11

\section{Document Preview \& Processing}
\label{sec:feature-document-preview}

% Content from PROJECT_REPORT.md Section 7.12

\section{Progressive Web App (PWA)}
\label{sec:feature-pwa}

% Content from PROJECT_REPORT.md Section 7.13

\section{Performance Optimization \& Security}
\label{sec:feature-performance}

% Content from PROJECT_REPORT.md Section 7.14

\section{SEO \& Accessibility}
\label{sec:feature-seo}

% Content from PROJECT_REPORT.md Section 7.15

% NOTE: Each section above should be expanded with detailed content
% from the corresponding section in PROJECT_REPORT.md
% For full implementation, copy the markdown content and convert
% to LaTeX formatting
