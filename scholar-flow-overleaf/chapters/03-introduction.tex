\chapter{Introduction}
\label{ch:introduction}

\section{Project Overview}
\label{sec:project-overview}

\projectname{} is an AI-powered research paper collaboration platform that enables researchers and students to manage, organize, and collaborate on academic papers efficiently. The system provides intelligent paper management, AI-powered insights, team collaboration features, and citation management.

\subsection{Problem Statement}

Academic researchers face several challenges:

\begin{itemize}[leftmargin=*,topsep=5pt,itemsep=3pt]
    \item \textbf{Information Overload}: Difficulty tracking and organizing vast numbers of research papers
    \item \textbf{Collaboration Barriers}: Lack of centralized platforms for team paper management  
    \item \textbf{Manual Processes}: Time-consuming metadata entry and citation management
    \item \textbf{Limited AI Integration}: No intelligent assistance for paper analysis
    \item \textbf{Fragmented Tools}: Using multiple disconnected applications
\end{itemize}

\subsection{Solution Approach}

ScholarFlow provides an integrated solution:

\begin{itemize}[leftmargin=*,topsep=5pt,itemsep=3pt]
    \item \textbf{Centralized Repository}: Secure cloud storage with AWS S3
    \item \textbf{AI Integration}: Gemini AI for summarization and contextual chat
    \item \textbf{Team Workspaces}: Role-based access control and collaboration
    \item \textbf{Automated Workflows}: Metadata extraction and PDF processing
    \item \textbf{Modern Stack}: Next.js, PostgreSQL, Express.js, Redis
\end{itemize}
