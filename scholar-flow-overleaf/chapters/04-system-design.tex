\chapter{System Design}
\label{ch:system-design}

\section{Architectural Overview}
\label{sec:architectural-overview}

\projectname{} follows a modular, three-tier architecture consisting of a web client, an API layer, and a persistent data layer.
The system is implemented as a monorepo with a \textbf{Next.js 15} frontend and an \textbf{Express.js} backend.
Data is stored in \textbf{PostgreSQL} and accessed through \textbf{Prisma} using parameterized RAW query patterns where appropriate.
For file storage and large binary objects, the platform integrates \textbf{AWS S3}.

The overall design aims to ensure (i) secure authentication and authorization, (ii) scalable paper processing and retrieval, and (iii) extensibility for AI-driven capabilities such as summarization and semantic search.

\section{Context Diagram}
\label{sec:context-diagram}

The context diagram (Figure~\ref{fig:context-diagram}) presents \projectname{} as a single system boundary and highlights the external actors and external services that interact with the platform.
The primary users are guest users (unauthenticated) and authenticated users working individually or within collaborative workspaces.
External services include OAuth providers for federated login, an email service for verification and password reset flows, Stripe for subscription billing, an AI provider for summarization/insights, and AWS S3 for object storage.

\begin{figure}[H]
    \centering
    \includegraphics[width=0.95\textwidth]{images/diagrams/context_diagram.jpeg}
    \caption{ScholarFlow Context Diagram}
    \label{fig:context-diagram}
\end{figure}

\section{Use Case Diagram and Descriptive Forms}
\label{sec:use-case-diagram}

The core feature use cases are captured in four focused diagrams (Figures~\ref{fig:use-case-diagram-1}--\ref{fig:use-case-diagram-4}) that highlight distinct interaction scopes.
To support academic clarity and traceability, representative use cases are specified using structured descriptive forms.

\begin{figure}[H]
    \centering
    \includegraphics[width=0.95\textwidth]{images/diagrams/use_case_1.jpeg}
    \caption{ScholarFlow Core Use Case Diagram (1)}
    \label{fig:use-case-diagram-1}
\end{figure}

\begin{figure}[H]
    \centering
    \includegraphics[width=0.95\textwidth]{images/diagrams/use_case_2.jpeg}
    \caption{ScholarFlow Core Use Case Diagram (2)}
    \label{fig:use-case-diagram-2}
\end{figure}

\begin{figure}[H]
    \centering
    \includegraphics[width=0.95\textwidth]{images/diagrams/use_case_3.jpeg}
    \caption{ScholarFlow Core Use Case Diagram (3)}
    \label{fig:use-case-diagram-3}
\end{figure}

\begin{figure}[H]
    \centering
    \includegraphics[width=0.95\textwidth]{images/diagrams/use_case_4.jpeg}
    \caption{ScholarFlow Core Use Case Diagram (4)}
    \label{fig:use-case-diagram-4}
\end{figure}

\subsection{Use Case Descriptions}
\label{subsec:use-case-descriptions}

\renewcommand{\arraystretch}{1.15}

\begin{table}[H]
    \centering
    \small
    \begin{tabularx}{\textwidth}{|l|X|}
        \hline
        \textbf{Use Case ID}                 & UC-01                                                                                                                                                                                           \\
        \hline
        \textbf{Name}                        & Register / Login                                                                                                                                                                                \\
        \hline
        \textbf{Primary Actor}               & Guest User                                                                                                                                                                                      \\
        \hline
        \textbf{Goal}                        & Create an account or authenticate to access the dashboard and workspace features.                                                                                                               \\
        \hline
        \textbf{Preconditions}               & The user has access to the web application; authentication service endpoints are available.                                                                                                     \\
        \hline
        \textbf{Main Success Scenario}       & (1) User selects register or login. (2) User provides credentials or chooses OAuth. (3) System validates input and issues a session/JWT. (4) User is redirected to the authenticated dashboard. \\
        \hline
        \textbf{Alternative/Exception Flows} & Invalid credentials; unverified email; OAuth failure; rate limiting on repeated attempts.                                                                                                       \\
        \hline
        \textbf{Postconditions}              & A valid authenticated session is established; user profile becomes accessible.                                                                                                                  \\
        \hline
    \end{tabularx}
    \caption{Use Case Description: Register / Login}
    \label{tab:uc-01}
\end{table}

\begin{table}[H]
    \centering
    \small
    \begin{tabularx}{\textwidth}{|l|X|}
        \hline
        \textbf{Use Case ID}                 & UC-02                                                                                                                                                                                                                                                               \\
        \hline
        \textbf{Name}                        & Upload Paper                                                                                                                                                                                                                                                        \\
        \hline
        \textbf{Primary Actor}               & Authenticated User                                                                                                                                                                                                                                                  \\
        \hline
        \textbf{Goal}                        & Upload a research paper to a workspace and store it reliably for later retrieval and processing.                                                                                                                                                                    \\
        \hline
        \textbf{Preconditions}               & The user is authenticated and has access to a target workspace.                                                                                                                                                                                                     \\
        \hline
        \textbf{Main Success Scenario}       & (1) User selects a file and submits an upload request. (2) System validates file type/size and metadata. (3) System stores the file in object storage and persists paper records in the database. (4) System returns a success response and updates the paper list. \\
        \hline
        \textbf{Alternative/Exception Flows} & Unsupported file type; upload interrupted; storage failure; insufficient workspace permissions.                                                                                                                                                                     \\
        \hline
        \textbf{Postconditions}              & Paper and file metadata are recorded; the file is available for preview and further processing.                                                                                                                                                                     \\
        \hline
    \end{tabularx}
    \caption{Use Case Description: Upload Paper}
    \label{tab:uc-02}
\end{table}

\begin{table}[H]
    \centering
    \small
    \begin{tabularx}{\textwidth}{|l|X|}
        \hline
        \textbf{Use Case ID}                 & UC-03                                                                                                                                                                                             \\
        \hline
        \textbf{Name}                        & Create Workspace and Invite Members                                                                                                                                                               \\
        \hline
        \textbf{Primary Actor}               & Workspace Owner                                                                                                                                                                                   \\
        \hline
        \textbf{Goal}                        & Create a collaborative workspace and invite other users for team-based paper management.                                                                                                          \\
        \hline
        \textbf{Preconditions}               & The user is authenticated; invitation email delivery is configured.                                                                                                                               \\
        \hline
        \textbf{Main Success Scenario}       & (1) Owner creates a workspace. (2) Owner submits member invitations. (3) System records membership/invitations and sends emails. (4) Invited users accept and become members with assigned roles. \\
        \hline
        \textbf{Alternative/Exception Flows} & Email delivery failure; invitee already a member; invalid email; role restrictions.                                                                                                               \\
        \hline
        \textbf{Postconditions}              & Workspace is created; membership records reflect the invited and accepted users.                                                                                                                  \\
        \hline
    \end{tabularx}
    \caption{Use Case Description: Create Workspace and Invite Members}
    \label{tab:uc-03}
\end{table}

\begin{table}[H]
    \centering
    \small
    \begin{tabularx}{\textwidth}{|l|X|}
        \hline
        \textbf{Use Case ID}                 & UC-04                                                                                                                                                                \\
        \hline
        \textbf{Name}                        & Create Collection and Add Paper                                                                                                                                      \\
        \hline
        \textbf{Primary Actor}               & Collection Owner                                                                                                                                                     \\
        \hline
        \textbf{Goal}                        & Organize papers into collections for structured reading and collaboration.                                                                                           \\
        \hline
        \textbf{Preconditions}               & The user is authenticated and has appropriate permissions within the workspace.                                                                                      \\
        \hline
        \textbf{Main Success Scenario}       & (1) User creates a collection with name/description. (2) User selects one or more papers. (3) System links papers to the collection and updates the collection view. \\
        \hline
        \textbf{Alternative/Exception Flows} & Duplicate collection names; paper not found; permission denied; invalid request payload.                                                                             \\
        \hline
        \textbf{Postconditions}              & A collection record exists and includes linked papers via the join table.                                                                                            \\
        \hline
    \end{tabularx}
    \caption{Use Case Description: Create Collection and Add Paper}
    \label{tab:uc-04}
\end{table}

\begin{table}[H]
    \centering
    \small
    \begin{tabularx}{\textwidth}{|l|X|}
        \hline
        \textbf{Use Case ID}                 & UC-05                                                                                                                                                                                                      \\
        \hline
        \textbf{Name}                        & Generate AI Summary                                                                                                                                                                                        \\
        \hline
        \textbf{Primary Actor}               & Authenticated User                                                                                                                                                                                         \\
        \hline
        \textbf{Goal}                        & Obtain an AI-generated summary of a selected paper to accelerate comprehension.                                                                                                                            \\
        \hline
        \textbf{Preconditions}               & The paper is uploaded and accessible; the AI provider integration is available.                                                                                                                            \\
        \hline
        \textbf{Main Success Scenario}       & (1) User requests a summary from the paper detail view. (2) System retrieves the paper text/chunks. (3) System calls the AI provider with appropriate prompts. (4) System stores and displays the summary. \\
        \hline
        \textbf{Alternative/Exception Flows} & AI provider error; rate limits; paper content unavailable; user plan does not include the feature.                                                                                                         \\
        \hline
        \textbf{Postconditions}              & A summary artifact is stored and can be reused without recomputation.                                                                                                                                      \\
        \hline
    \end{tabularx}
    \caption{Use Case Description: Generate AI Summary}
    \label{tab:uc-05}
\end{table}

\section{Data Flow Diagram (DFD)}
\label{sec:dfd}

The Data Flow Diagram (Figure~\ref{fig:dfd-diagram}) illustrates how data moves through \projectname{} processes and stores.
At a high level, the system receives user requests through the web client, performs authentication and authorization in the API layer, and persists results in the database and object storage.
The main processes include session management, workspace and membership operations, paper upload and processing, AI-driven analysis, and subscription management.

\begin{figure}[H]
    \centering
    \includegraphics[width=0.98\textwidth]{images/diagrams/dfd_diagram.png}
    \caption{ScholarFlow Data Flow Diagram (DFD)}
    \label{fig:dfd-diagram}
\end{figure}

\subsection{DFD Notes (Process and Data Store Overview)}
\begin{itemize}[leftmargin=*,topsep=5pt,itemsep=3pt]
    \item \textbf{Processes}: authentication/session handling, workspace and collection management, paper ingestion and metadata extraction, AI summarization/insights, and billing operations.
    \item \textbf{Data Stores}: PostgreSQL stores structured entities (users, workspaces, papers, collections, memberships, AI artifacts, billing records), while AWS S3 stores uploaded documents and derived file assets.
    \item \textbf{External Interfaces}: OAuth providers and email services support identity verification workflows; Stripe manages subscription events; AI providers perform inference for summaries and insights.
\end{itemize}

\section{Activity Diagram}
\label{sec:activity-diagram}

The activity diagram (Figure~\ref{fig:activity-diagram}) captures the operational workflow for paper upload, validation, persistence, and completion.
It emphasizes validation points (file checks, field validation, authorization) and highlights the separation between user actions and backend processing responsibilities.

\begin{figure}[H]
    \centering
    \includegraphics[width=0.92\textwidth]{images/diagrams/activity_diagram.jpg}
    \caption{Activity Diagram for Paper Upload and Persistence}
    \label{fig:activity-diagram}
\end{figure}

\section{Sequence Diagram}

The sequence diagram (Figure~\ref{fig:sequence-diagram}) captures the main interaction flow between the web client, API, database, storage, and AI provider.

\begin{figure}[H]
    \centering
    \includegraphics[width=0.95\textwidth]{images/diagrams/squence_diagram.png}
    \caption{ScholarFlow Main Sequence Diagram}
    \label{fig:sequence-diagram}
\end{figure}

\section{State Diagram}

The state diagram (Figure~\ref{fig:state-diagram}) models the lifecycle of papers from upload and processing through moderation and publication.

\begin{figure}[H]
    \centering
    \includegraphics[width=0.8\textwidth]{images/diagrams/state.png}
    \caption{Paper Lifecycle State Diagram}
    \label{fig:state-diagram}
\end{figure}

\section{Class Diagram}

The class diagram (Figure~\ref{fig:class-diagram}) summarizes the core domain entities and their relationships derived from the Prisma data model.

\begin{figure}[H]
    \centering
    \includegraphics[width=0.95\textwidth]{images/diagrams/class_diagram.png}
    \caption{ScholarFlow Domain Class Diagram}
    \label{fig:class-diagram}
\end{figure}

\section{CRC Cards}

CRC (Class-Responsibility-Collaborator) cards capture the essential responsibilities and collaborators for primary domain classes.

\subsection{User}

\begin{table}[H]
    \centering
    \small
    \begin{tabularx}{\textwidth}{|l|X|}
        \hline
        \textbf{Responsibility}          & \textbf{Collaborator} \\
        \hline
        Authenticate and manage session  & Account, Session      \\
        \hline
        Create and own workspaces        & Workspace             \\
        \hline
        Upload and manage papers         & Paper, PaperFile      \\
        \hline
        Join workspaces as member        & WorkspaceMember       \\
        \hline
        Manage payments and subscription & Subscription, Payment \\
        \hline
    \end{tabularx}
    \caption{CRC Card: User}
    \label{tab:crc-user}
\end{table}

\subsection{Workspace}

\begin{table}[H]
    \centering
    \small
    \begin{tabularx}{\textwidth}{|l|X|}
        \hline
        \textbf{Responsibility}              & \textbf{Collaborator} \\
        \hline
        Container for papers and collections & Paper, Collection     \\
        \hline
        Manage members and roles             & WorkspaceMember, User \\
        \hline
        Handle invitations                   & WorkspaceInvitation   \\
        \hline
        Track activity within scope          & ActivityLog           \\
        \hline
    \end{tabularx}
    \caption{CRC Card: Workspace}
    \label{tab:crc-workspace}
\end{table}

\subsection{Paper}

\begin{table}[H]
    \centering
    \small
    \begin{tabularx}{\textwidth}{|l|X|}
        \hline
        \textbf{Responsibility}              & \textbf{Collaborator}      \\
        \hline
        Store metadata and content reference & PaperFile, Workspace       \\
        \hline
        Track processing status              & PaperProcessingStatus      \\
        \hline
        Provide chunks for AI analysis       & PaperChunk                 \\
        \hline
        Store AI-generated insights          & AISummary, AIInsightThread \\
        \hline
        Maintain citation links              & Citation                   \\
        \hline
    \end{tabularx}
    \caption{CRC Card: Paper}
    \label{tab:crc-paper}
\end{table}

\subsection{Collection}

\begin{table}[H]
    \centering
    \small
    \begin{tabularx}{\textwidth}{|l|X|}
        \hline
        \textbf{Responsibility}        & \textbf{Collaborator}  \\
        \hline
        Group papers logically         & Paper, CollectionPaper \\
        \hline
        Manage access permissions      & CollectionMember, User \\
        \hline
        Belong to a specific workspace & Workspace              \\
        \hline
    \end{tabularx}
    \caption{CRC Card: Collection}
    \label{tab:crc-collection}
\end{table}

\section{ERD and Relational Schema Diagrams}
\label{sec:erd-and-schema}

The database design of \projectname{} is centered around the entities required for authentication, collaborative workspaces, paper management, and AI features.
The ERD (Figure~\ref{fig:system-erd}) visualizes the conceptual model and relationships, while the schema diagram (Figure~\ref{fig:system-schema}) represents the logical relational structure used in the PostgreSQL database.

\begin{figure}[H]
    \centering
    \includegraphics[width=0.98\textwidth]{images/diagrams/ERD.png}
    \caption{ScholarFlow ERD (Conceptual Data Model)}
    \label{fig:system-erd}
\end{figure}

\begin{figure}[H]
    \centering
    \includegraphics[width=0.98\textwidth]{images/diagrams/schema.png}
    \caption{ScholarFlow Relational Schema Diagram}
    \label{fig:system-schema}
\end{figure}

\subsection{Core Entities and Relationships}
\begin{itemize}[leftmargin=*,topsep=5pt,itemsep=3pt]
    \item \textbf{User, Account, Session, UserToken}: authentication and OAuth linkage.
    \item \textbf{Workspace, WorkspaceMember}: collaborative boundaries and role-based membership.
    \item \textbf{Paper, PaperFile, PaperChunk}: paper storage, metadata, and chunking for downstream AI operations.
    \item \textbf{Collection, CollectionPaper, CollectionMember}: grouping of papers and controlled sharing.
    \item \textbf{AISummary}: persisted AI outputs associated with papers.
\end{itemize}
