\chapter{Motivation}
\label{ch:motivation}

\section{Problem Context}
\label{sec:problem-context}

Academic researchers face several critical challenges that motivated the development of ScholarFlow:

\begin{itemize}[leftmargin=*,topsep=5pt,itemsep=4pt]
    \item \textbf{Fragmented Tools}: Researchers juggle multiple disconnected applications (Mendeley for references, Dropbox for storage, Slack for collaboration), leading to reduced productivity
    
    \item \textbf{Poor Organization}: Traditional file systems fail to scale for large paper collections, resulting in difficulty locating specific papers
    
    \item \textbf{Limited Collaboration}: Existing tools lack real-time collaboration features and role-based access control
    
    \item \textbf{Manual Processes}: Time-consuming metadata entry and citation management with high error rates
    
    \item \textbf{No AI Integration}: Lack of intelligent assistance for paper summarization and analysis
    
    \item \textbf{High Costs}: Commercial research tools are expensive for students and individual researchers
\end{itemize}

\subsection{Solution Goals}

\begin{itemize}[leftmargin=*,topsep=5pt,itemsep=3pt]
    \item Provide unified platform for paper management and collaboration
    \item Automate metadata extraction and citation management
    \item Integrate AI for intelligent paper analysis
    \item Offer affordable pricing with generous free tier
    \item Enable team collaboration with role-based permissions
\end{itemize}
