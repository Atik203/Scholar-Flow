\chapter{Introduction}

% ============================================
% 1.1 PURPOSE
% ============================================
\section{Purpose}

This Software Requirements Specification (SRS) document provides a comprehensive description of the \textbf{ScholarFlow} platform, an AI-powered research paper collaboration hub designed to revolutionize how academic researchers manage, organize, and collaborate on research papers. This document is intended for:

\begin{itemize}[leftmargin=*]
    \item \textbf{Development Team}: To understand the complete system requirements and implementation guidelines
    \item \textbf{Project Stakeholders}: To review and validate the proposed features and functionalities
    \item \textbf{Academic Reviewers}: To assess the project scope and technical feasibility
    \item \textbf{End Users}: To understand the capabilities and benefits of the ScholarFlow platform
    \item \textbf{Quality Assurance Team}: To design comprehensive test plans and validation criteria
\end{itemize}

The primary objectives of this document are to:
\begin{enumerate}
    \item Define the functional and non-functional requirements of the ScholarFlow system
    \item Establish clear system boundaries and scope
    \item Document user roles, permissions, and interaction patterns
    \item Specify technical architecture and integration points
    \item Provide a foundation for project planning and resource allocation
\end{enumerate}

% ============================================
% 1.2 PROJECT SCOPE
% ============================================
\section{Project Scope}

\subsection{Product Perspective}

ScholarFlow is a comprehensive web-based platform that addresses critical gaps in the academic research workflow. The system integrates multiple components to create a unified research management ecosystem:

\begin{figure}[H]
    \centering
    \begin{tikzpicture}[
        node distance=2cm,
        box/.style={rectangle, draw=primaryblue, thick, fill=lightgray, rounded corners, minimum width=3cm, minimum height=1cm, text centered, font=\small},
        arrow/.style={->, >=stealth, thick, draw=primaryblue}
    ]
        % Central node
        \node[box, fill=primaryblue!20] (core) at (0,0) {\textbf{ScholarFlow Core}};
        
        % Surrounding nodes
        \node[box] (auth) at (-3,3) {Authentication \& Security};
        \node[box] (paper) at (3,3) {Paper Management};
        \node[box] (ai) at (-3,0) {AI Services};
        \node[box] (collab) at (3,0) {Collaboration};
        \node[box] (analytics) at (-3,-3) {Analytics \& Reports};
        \node[box] (billing) at (3,-3) {Subscription \& Billing};
        
        % Arrows
        \draw[arrow] (auth) -- (core);
        \draw[arrow] (paper) -- (core);
        \draw[arrow] (ai) -- (core);
        \draw[arrow] (collab) -- (core);
        \draw[arrow] (analytics) -- (core);
        \draw[arrow] (billing) -- (core);
    \end{tikzpicture}
    \caption{ScholarFlow System Architecture Overview}
    \label{fig:system-overview}
\end{figure}

\subsection{Product Features (High-Level)}

The ScholarFlow platform encompasses the following major feature categories:

\begin{table}[H]
\centering
\begin{tabularx}{\textwidth}{|l|X|}
\hline
\textbf{Feature Category} & \textbf{Description} \\
\hline
\textbf{User Management} & 
Multi-provider authentication (Email/Password, Google OAuth, GitHub OAuth), role-based access control, profile management, and session security \\
\hline
\textbf{Paper Management} & 
PDF upload and storage, metadata extraction, version history, advanced search and filtering, tag management, and file organization \\
\hline
\textbf{Rich Text Editor} & 
TipTap-based WYSIWYG editor with auto-save, markdown support, collaborative editing, comments, and mentions \\
\hline
\textbf{AI Integration} & 
Multi-provider AI service (Google Gemini, OpenAI), context-aware chat, paper summarization, citation generation, and recommendation engine \\
\hline
\textbf{Collaboration} & 
Workspace creation, 5-tier role system (Owner, Admin, Editor, Commenter, Viewer), real-time collaboration, and sharing permissions \\
\hline
\textbf{Analytics Dashboard} & 
Personal analytics, workspace analytics, admin system metrics, usage tracking, and export capabilities \\
\hline
\textbf{Subscription System} & 
Freemium model with Free, Pro, and Team tiers, Stripe integration, quota management, and billing history \\
\hline
\end{tabularx}
\caption{Major Feature Categories in ScholarFlow}
\label{table:feature-categories}
\end{table}

\subsection{Target User Demographics}

Based on comprehensive survey data from 29 academic researchers across 10 universities, ScholarFlow targets the following primary user segments:

\begin{itemize}[leftmargin=*]
    \item \textbf{Primary Segment (86.2\%)}: Undergraduate students conducting research projects, particularly in Computer Science and IT fields
    \item \textbf{Secondary Segment (10.4\%)}: Graduate students (Masters and PhD) managing larger research portfolios
    \item \textbf{Tertiary Segment (3.4\%)}: Faculty members, research assistants, and academic professionals
\end{itemize}

\textbf{Key User Characteristics:}
\begin{itemize}
    \item \textbf{Age Range}: 75.9\% between 22-25 years old (Gen-Z, mobile-first generation)
    \item \textbf{Academic Level}: 58.6\% are 3rd-year undergraduates at peak research intensity
    \item \textbf{Technical Proficiency}: 67\%+ from CS/IT backgrounds (tech-savvy early adopters)
    \item \textbf{Geographic Focus}: Bangladesh market with UIU concentration (47.4\% of respondents)
    \item \textbf{Current Pain Points}: 37.9\% cite "lack of proper tools" as primary research challenge
\end{itemize}

\subsection{Operating Environment}

ScholarFlow operates in a modern cloud-based infrastructure with the following technical environment:

\begin{table}[H]
\centering
\begin{tabularx}{\textwidth}{|l|X|}
\hline
\textbf{Component} & \textbf{Technology Stack} \\
\hline
\textbf{Frontend} & 
Next.js 15 (React 19), TypeScript, Tailwind CSS, ShadCN UI components, Redux Toolkit Query for state management \\
\hline
\textbf{Backend} & 
Node.js with Express.js, TypeScript, RESTful API architecture, JWT-based authentication \\
\hline
\textbf{Database} & 
PostgreSQL 14+ with Prisma ORM, type-safe queries, migrations support \\
\hline
\textbf{File Storage} & 
AWS S3 for scalable object storage, CloudFront CDN for content delivery \\
\hline
\textbf{AI Services} & 
Google Gemini 2.5-flash-lite (primary), OpenAI GPT-4o-mini (fallback), multi-provider architecture \\
\hline
\textbf{Payment Processing} & 
Stripe API for subscription management, webhook handlers for real-time events \\
\hline
\textbf{Deployment} & 
Vercel for frontend hosting, serverless functions, edge network distribution \\
\hline
\textbf{Development Tools} & 
Yarn Berry (v4), Turborepo monorepo, ESLint, Prettier, Jest for testing \\
\hline
\end{tabularx}
\caption{ScholarFlow Technology Stack}
\label{table:tech-stack}
\end{table}

\subsection{Design and Implementation Constraints}

The following constraints influence the design and implementation of ScholarFlow:

\begin{enumerate}
    \item \textbf{Performance Constraints}:
    \begin{itemize}
        \item Page load time must be under 3 seconds on 4G connections
        \item API response time should not exceed 500ms for standard queries
        \item Large file uploads (up to 50MB per paper) must be handled efficiently
        \item Real-time collaboration features require WebSocket support
    \end{itemize}
    
    \item \textbf{Security Constraints}:
    \begin{itemize}
        \item All data transmission must use HTTPS/TLS encryption
        \item Password storage requires bcrypt hashing with minimum 10 rounds
        \item JWT tokens must expire within 24 hours
        \item File uploads must be validated and sanitized to prevent malicious content
    \end{itemize}
    
    \item \textbf{Regulatory Constraints}:
    \begin{itemize}
        \item GDPR compliance for European users (data privacy, right to deletion)
        \item CCPA compliance for California users
        \item Data residency requirements for educational institutions
        \item Accessibility standards (WCAG 2.1 Level AA)
    \end{itemize}
    
    \item \textbf{Resource Constraints}:
    \begin{itemize}
        \item Development team: 4 members (1 project leader + 3 developers)
        \item Budget limitations for third-party API costs (AI, storage, payment processing)
        \item Academic timeline constraints (semester-based development)
    \end{itemize}
    
    \item \textbf{Technical Constraints}:
    \begin{itemize}
        \item Browser compatibility: Chrome 90+, Firefox 88+, Safari 14+, Edge 90+
        \item Mobile responsiveness required for all features
        \item Offline functionality for core reading and editing features
        \item Database query optimization to prevent N+1 problems
    \end{itemize}
\end{enumerate}

% ============================================
% 1.3 DEFINITIONS, ACRONYMS, AND ABBREVIATIONS
% ============================================
\section{Definitions, Acronyms, and Abbreviations}

\subsection{Definitions}

\begin{description}[leftmargin=3cm, style=nextline]
    \item[\textbf{Workspace}] A collaborative environment where multiple users can organize and share research papers, notes, and resources under shared permissions and quotas.
    
    \item[\textbf{Paper}] A research document (typically PDF format) uploaded to ScholarFlow, including its metadata, annotations, notes, and version history.
    
    \item[\textbf{Collection}] A user-defined grouping of papers organized by topic, project, or any custom criteria for better organization.
    
    \item[\textbf{Rich Text Note}] A formatted document created using the TipTap editor, supporting markdown, mentions, comments, and collaborative editing.
    
    \item[\textbf{AI Context}] The accumulated conversation history and referenced papers used by the AI chat assistant to provide contextually relevant responses.
    
    \item[\textbf{Subscription Tier}] The level of service (Free, Pro, Team) that determines user quotas, feature access, and billing amounts.
    
    \item[\textbf{Role-Based Access}] A permission system with 5 levels (Owner, Admin, Editor, Commenter, Viewer) controlling user capabilities within a workspace.
\end{description}

\subsection{Acronyms}

\begin{table}[H]
\centering
\begin{tabularx}{\textwidth}{|l|X|}
\hline
\textbf{Acronym} & \textbf{Full Form} \\
\hline
API & Application Programming Interface \\
\hline
AWS & Amazon Web Services \\
\hline
CDN & Content Delivery Network \\
\hline
CORS & Cross-Origin Resource Sharing \\
\hline
CRUD & Create, Read, Update, Delete \\
\hline
DBMS & Database Management System \\
\hline
ERD & Entity Relationship Diagram \\
\hline
GDPR & General Data Protection Regulation \\
\hline
HTTPS & Hypertext Transfer Protocol Secure \\
\hline
JWT & JSON Web Token \\
\hline
MVP & Minimum Viable Product \\
\hline
OAuth & Open Authorization \\
\hline
ORM & Object-Relational Mapping \\
\hline
PDF & Portable Document Format \\
\hline
REST & Representational State Transfer \\
\hline
S3 & Simple Storage Service (AWS) \\
\hline
SRS & Software Requirements Specification \\
\hline
SSL/TLS & Secure Sockets Layer / Transport Layer Security \\
\hline
UI/UX & User Interface / User Experience \\
\hline
WCAG & Web Content Accessibility Guidelines \\
\hline
WYSIWYG & What You See Is What You Get \\
\hline
\end{tabularx}
\caption{List of Acronyms}
\label{table:acronyms}
\end{table}

% ============================================
% 1.4 REFERENCES
% ============================================
\section{References}

\begin{enumerate}
    \item IEEE Std 830-1998, IEEE Recommended Practice for Software Requirements Specifications
    \item Next.js 15 Documentation: \url{https://nextjs.org/docs}
    \item Express.js API Reference: \url{https://expressjs.com/en/5x/api.html}
    \item PostgreSQL 14 Documentation: \url{https://www.postgresql.org/docs/14/}
    \item Prisma ORM Documentation: \url{https://www.prisma.io/docs}
    \item AWS S3 Developer Guide: \url{https://docs.aws.amazon.com/s3/}
    \item Google Gemini API Documentation: \url{https://ai.google.dev/docs}
    \item OpenAI API Reference: \url{https://platform.openai.com/docs}
    \item Stripe API Documentation: \url{https://stripe.com/docs/api}
    \item TipTap Editor Documentation: \url{https://tiptap.dev/docs}
    \item ScholarFlow Feasibility Survey Results (29 responses, November-December 2025)
    \item ScholarFlow GitHub Repository: \url{https://github.com/Atik203/Scholar-Flow}
\end{enumerate}

% ============================================
% 1.5 OVERVIEW OF DOCUMENT
% ============================================
\section{Overview of Document}

This Software Requirements Specification is organized into the following chapters:

\begin{description}[leftmargin=3cm, style=nextline]
    \item[\textbf{Chapter 1: Introduction}] 
    Provides the purpose, scope, definitions, and references for the ScholarFlow system. Establishes the context and objectives of the project.
    
    \item[\textbf{Chapter 2: System Study \& Information Gathering}] 
    Documents the survey methodology, demographic analysis, current tool landscape, market validation, and pain points identified through comprehensive user research with 29 academic researchers.
    
    \item[\textbf{Chapter 3: System Analysis}] 
    Presents feasibility study results, SWOT analysis, competitive analysis, feature prioritization matrix, and strategic recommendations based on survey data and market research.
    
    \item[\textbf{Chapter 4: System Design}] 
    Details the system architecture, database schema, Entity-Relationship Diagrams (ERD), API design, and technical implementation specifications.
    
    \item[\textbf{Chapter 5: System Implementation}] 
    Describes the development methodology, technology stack integration, code structure, deployment pipeline, and testing strategies.
    
    \item[\textbf{Chapter 6: Testing \& Validation}] 
    Outlines the comprehensive testing approach including unit tests, integration tests, end-to-end tests, and user acceptance testing results.
    
    \item[\textbf{Chapter 7: Results \& Screenshots}] 
    Showcases the implemented system with annotated screenshots, feature demonstrations, and user interface walkthroughs.
    
    \item[\textbf{Chapter 8: Limitations \& Constraints}] 
    Identifies current system limitations, known issues, scope boundaries, and areas requiring future enhancement.
    
    \item[\textbf{Chapter 9: Future Work}] 
    Proposes roadmap for Phase 2 and Phase 3 features, scalability improvements, and long-term product evolution.
    
    \item[\textbf{Chapter 10: Conclusion}] 
    Summarizes project achievements, lessons learned, and overall assessment of ScholarFlow's success in addressing academic research needs.
\end{description}
