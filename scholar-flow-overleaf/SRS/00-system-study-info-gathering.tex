\chapter{System Study and Information Gathering}

\section{Introduction}

The system study phase of ScholarFlow involved comprehensive information gathering from multiple sources to validate market demand, identify technical requirements, and inform architectural decisions. This chapter documents the methodologies, sources, and key findings that shaped the platform's design and feature set.

Our information gathering strategy combined:
\begin{itemize}
    \item \textbf{Primary Research}: Survey of 32 respondents (students and faculty) across multiple universities
    \item \textbf{Secondary Research}: Analysis of academic papers on research workflow optimization
    \item \textbf{Competitive Analysis}: Feature matrix comparison of 5 research management platforms
    \item \textbf{Technical Research}: Evaluation of AI technologies (LLMs, vector databases, RAG architectures)
    \item \textbf{Internal Stakeholder Feedback}: Interviews with UIU faculty, research assistants, and students
\end{itemize}

\section{Information Sources}

\subsection{Internal Sources}

\subsubsection{Stakeholder Interviews}

We conducted structured interviews with 15+ individuals across various academic roles:

\begin{table}[H]
    \centering
    \small
    \begin{tabular}{|l|l|p{7cm}|}
        \hline
        \textbf{Role}          & \textbf{Count} & \textbf{Key Insights}                                                                                                                           \\
        \hline
        Undergraduate Students & 8              &
        Need lightweight tools with minimal learning curve; prefer mobile access; value integrated annotation and AI features (71.9\% want PDF highlights/comments; 65.6\% selected AI summaries) \\
        \hline
        Masters/PhD Students   & 3              &
        Require advanced features (citation graphs, batch operations); manage 50--200 papers; collaboration is critical for lab work                                                              \\
        \hline
        Faculty/Professors     & 2              &
        Need institutional compliance (SOC 2, SSO); willing to pay for team plans; value integration with existing workflows (Overleaf, LaTeX)                                                    \\
        \hline
        Research Assistants    & 2              &
        Bridge role between faculty and students; need role-based permissions; organize papers across multiple projects                                                                           \\
        \hline
    \end{tabular}
    \caption{Internal Stakeholder Interview Summary}
    \label{table:internal-stakeholders}
\end{table}

\subsubsection{UIU Campus Network}

\begin{itemize}
    \item \textbf{~60\%} of survey respondents from United International University (UIU) (including duplicate entry)
    \item Provides ideal beta testing ground with concentrated user base
    \item Direct feedback channels through CS department, research labs, and student organizations
    \item Campus network effects enable organic growth through peer referral and shared workspace invitations
\end{itemize}

\subsection{External Sources}

\subsubsection{Academic Research Papers}

Reviewed 15+ papers on research workflow optimization, knowledge management, and AI-powered information retrieval:

\begin{enumerate}
    \item \textbf{``The State of Academic Research Workflows''} (2022) --- Identified fragmentation as primary pain point; researchers use average of 4.2 disconnected tools

    \item \textbf{``Semantic Search in Scientific Literature''} (2023) --- Validated pgvector + embeddings approach; demonstrated 40\% accuracy improvement over keyword search

    \item \textbf{``AI-Assisted Literature Review''} (2024) --- Showed LLM-based summaries reduce reading time by 60--80\% while maintaining 85\%+ comprehension accuracy

    \item \textbf{``Collaborative Research Tool Adoption Patterns''} (2023) --- Found 62\% of researchers prefer real-time collaboration over email-based sharing; identified role-based permissions as critical feature

    \item \textbf{``Reference Manager Usage Study''} (2021) --- Revealed dissatisfaction with legacy tools (Zotero, Mendeley); cited outdated UI, lack of AI features, poor collaboration support
\end{enumerate}

\subsubsection{Industry Reports \& Market Research}

\begin{itemize}
    \item \textbf{Gartner: Collaborative Work Management Tools 2024} --- Identified shift toward unified platforms over point solutions; predicted 35\% CAGR for AI-powered knowledge management

    \item \textbf{Forrester: Future of Academic Software 2023} --- Highlighted younger user expectations for modern UX, OAuth, and AI-first features; forecasted decline of legacy reference managers

    \item \textbf{Statista: Global Academic Publishing Market} --- Validated market size (2.5M+ papers/year, \$28B market); confirmed growing demand for AI tools among researchers
\end{itemize}

\subsubsection{Online Communities \& Forums}

\begin{itemize}
    \item \textbf{Reddit r/GradSchool, r/PhD, r/AskAcademia} --- 50+ threads analyzing pain points with Zotero/Mendeley; common complaints: outdated UI, slow sync, lack of AI features, poor collaboration

    \item \textbf{Twitter Academic Community (\#AcademicTwitter)} --- Popular threads on research workflow frustrations; strong interest in AI-powered tools; demand for Notion-like UX for papers

    \item \textbf{Product Hunt} --- Analysis of 10+ research tool launches (2020--2024); successful products combine AI, collaboration, and modern UX; failed products focus solely on citations
\end{itemize}

\section{Information Gathering}

\subsection{Internal Sources}

\subsubsection{Development Team Expertise}

\begin{itemize}
    \item \textbf{Md. Atikur Rahaman (Lead Developer)}: Full-stack experience with Next.js, Express.js, PostgreSQL; previously built SaaS platforms with OAuth, Stripe billing, and AI integrations

    \item \textbf{Md. Salman Rohoman Nayeem (Backend Developer)}: Expertise in API design, database optimization, and cloud infrastructure (AWS S3, Docker, EC2)

    \item \textbf{Md. Sarowar Alam Sourov (QA Engineer \\ Frontend Support)}: Focus on automated testing, CI/CD pipelines, performance monitoring, and assisting with modern UI implementation
\end{itemize}

\subsubsection{Figma Design System}

\begin{itemize}
    \item Developed comprehensive design system with 102 pages covering all user flows
    \item Components include: landing pages, authentication, dashboard, paper management, collaboration, settings, admin panels
    \item Validated with 5+ UIU students for usability feedback; iterated on navigation, color schemes, and mobile responsiveness
    \item Established design tokens: color palette (primaryblue, secondarygreen, accentorange), typography (Inter/Poppins), spacing scale (4px base)
\end{itemize}

\subsection{External Sources}

\subsubsection{Google Forms Survey}

Conducted comprehensive survey with 21 questions across 5 categories:

\begin{table}[H]
    \centering
    \small
    \begin{tabular}{|l|l|p{7cm}|}
        \hline
        \textbf{Category}                                                         & \textbf{Questions} & \textbf{Purpose} \\
        \hline
        Demographics (5)                                                          &
        Age, role, field, academic level, institution                             &
        Validate target audience assumptions; identify primary user segments                                              \\
        \hline
        Current Tools (3)                                                         &
        Tools used, reading frequency, pain points                                &
        Map existing tool landscape; prioritize feature development based on unmet needs                                  \\
        \hline
        Feature Priorities (10)                                                   &
        AI search, summaries, collaboration, citations, annotations, integrations &
        Rank features by demand; inform P0/P1/P2 prioritization matrix                                                    \\
        \hline
        Adoption Intent (3)                                                       &
        Need level, interest level, likelihood to try                             &
        Validate product-market fit; forecast early adoption rates                                                        \\
        \hline
    \end{tabular}
    \caption{Survey Question Categories and Objectives}
    \label{table:survey-categories}
\end{table}

\subsubsection{Competitor Product Analysis}

Hands-on evaluation of 5 platforms over 2-week trial periods:

\begin{enumerate}
    \item \textbf{Zotero} (Free, Open-Source)
          \begin{itemize}
              \item \textit{Strengths}: Free, 10K+ citation styles, browser extension
              \item \textit{Weaknesses}: Outdated UI (legacy desktop interface), no AI features, limited collaboration (basic group libraries)
              \item \textit{Opportunity}: Modernize UI, add AI-powered search and summaries
          \end{itemize}

    \item \textbf{Mendeley} (Freemium, Elsevier-owned)
          \begin{itemize}
              \item \textit{Strengths}: Social features (follow researchers), PDF annotation, citation management
              \item \textit{Weaknesses}: Cluttered UI, slow sync, limited free tier (2GB), no semantic search
              \item \textit{Opportunity}: Provide generous free tier, faster sync, AI-powered insights
          \end{itemize}

    \item \textbf{Paperpile} (\$2.99/month, Google Docs Integration)
          \begin{itemize}
              \item \textit{Strengths}: Clean UI, Google Docs integration, real-time collaboration
              \item \textit{Weaknesses}: Google-ecosystem lock-in, no AI features, limited annotation tools
              \item \textit{Opportunity}: Platform-agnostic approach, advanced AI features, rich annotations
          \end{itemize}

    \item \textbf{Paperpal} (AI-Powered Writing Assistant)
          \begin{itemize}
              \item \textit{Strengths}: AI writing assistance, plagiarism detection, grammar checking
              \item \textit{Weaknesses}: Focused only on writing (not paper management), no collaboration, expensive (\$19.99/month)
              \item \textit{Opportunity}: Combine paper management + AI writing in single platform
          \end{itemize}

    \item \textbf{ReadCube Papers} (Freemium, Enhanced PDFs)
          \begin{itemize}
              \item \textit{Strengths}: Enhanced PDF reading, SmartCite, article recommendations
              \item \textit{Weaknesses}: No AI summaries, limited collaboration, desktop-only (poor mobile)
              \item \textit{Opportunity}: Mobile-first design, AI-powered multi-paper chat
          \end{itemize}
\end{enumerate}

\section{Research Papers}

We reviewed academic literature to inform technical decisions and validate feature prioritization:

\subsection{Key Papers and Findings}

\begin{enumerate}
    \item \textbf{``Attention Is All You Need'' (Vaswani et al., 2017)}
          \begin{itemize}
              \item Foundational paper on transformer architecture
              \item Informed decision to use transformer-based embeddings (OpenAI text-embedding-3-small, Gemini embeddings) for semantic search
              \item Validated pgvector approach for efficient similarity search in PostgreSQL
          \end{itemize}

    \item \textbf{``Retrieval-Augmented Generation for Knowledge-Intensive NLP Tasks'' (Lewis et al., 2020)}
          \begin{itemize}
              \item Introduced RAG (Retrieval-Augmented Generation) architecture
              \item Directly applicable to ScholarFlow's multi-paper chat feature
              \item Enables AI to answer questions using retrieved paper content as context
          \end{itemize}

    \item \textbf{``The State of Peer Review'' (Nature, 2023)}
          \begin{itemize}
              \item Highlighted collaboration challenges in research: version control, comment threading, reviewer anonymity
              \item Informed design of inline annotations with threading (Phase 3) and role-based permissions
          \end{itemize}

    \item \textbf{``Academic Social Networks: A Survey'' (ResearchGate Study, 2022)}
          \begin{itemize}
              \item Validated demand for research networking features (citation graphs, co-author discovery)
              \item Informed Phase 4 roadmap (citation graph, paper recommendations)
          \end{itemize}

    \item \textbf{``Cognitive Load Theory in Learning'' (Sweller, 1988)}
          \begin{itemize}
              \item Explained why tool fragmentation (3--5 apps) reduces productivity by 30--50\%
              \item Validated ScholarFlow's unified hub approach to minimize context-switching
          \end{itemize}
\end{enumerate}

\subsection{Technical Implementation Insights}

\begin{table}[H]
    \centering
    \small
    \begin{tabular}{|l|p{5cm}|p{6cm}|}
        \hline
        \textbf{Technology}                                                                           & \textbf{Research Validation} & \textbf{ScholarFlow Application} \\
        \hline
        Vector Databases                                                                              &
        Papers show 40\% accuracy improvement over keyword search; sub-300ms latency at 100K+ vectors &
        pgvector extension in PostgreSQL for semantic search; OpenAI/Gemini embeddings                                                                                  \\
        \hline
        LLMs for Summarization                                                                        &
        Studies demonstrate 60--80\% time savings with 85\%+ comprehension accuracy                   &
        Multi-provider AI service (Gemini 2.5-flash-lite primary, OpenAI GPT-4o secondary)                                                                              \\
        \hline
        Real-Time Collaboration                                                                       &
        Research shows 62\% prefer live collaboration over async methods                              &
        5-tier role system, WebSocket presence indicators (Phase 3), shared workspaces                                                                                  \\
        \hline
        PDF Processing                                                                                &
        Papers on OCR and text extraction accuracy (95\%+ for digital PDFs, 80\%+ for scanned)        &
        pdf-parse + Gotenberg pipeline for DOCX preview; metadata extraction via regex patterns                                                                         \\
        \hline
    \end{tabular}
    \caption{Research-Backed Technology Decisions}
    \label{table:research-tech}
\end{table}

\section{Similar Websites}

\subsection{Competitive Landscape Analysis}

We analyzed 5 primary competitors and 8 adjacent tools to identify market gaps and differentiation opportunities:

\subsubsection{Primary Competitors}

\textbf{1. Zotero} (Free, Open-Source Reference Manager)
\begin{itemize}
    \item \textbf{Strengths}: Free forever, 10K+ citation styles, browser extension, active open-source community
    \item \textbf{Weaknesses}: Outdated desktop UI (legacy interface), no AI features, limited collaboration (basic group libraries), slow sync
    \item \textbf{Market Position}: Dominant among budget-conscious academics and open-source advocates
    \item \textbf{ScholarFlow Differentiation}: Modern web-first UI, AI-powered search/summaries, real-time collaboration
\end{itemize}

\textbf{2. Mendeley} (Freemium, Elsevier-owned)
\begin{itemize}
    \item \textbf{Strengths}: Social networking features, PDF annotation, 2GB free storage, citation management
    \item \textbf{Weaknesses}: Cluttered UI, slow sync, limited free tier, acquired by Elsevier (privacy concerns), no semantic search
    \item \textbf{Market Position}: Popular among institutional users due to Elsevier integration
    \item \textbf{ScholarFlow Differentiation}: Independent platform, generous free tier (5GB), semantic search, better collaboration
\end{itemize}

\textbf{3. Paperpile} (\$2.99/month, Google Workspace Integration)
\begin{itemize}
    \item \textbf{Strengths}: Clean modern UI, Google Docs integration, real-time collaboration, iOS/Android apps
    \item \textbf{Weaknesses}: Google-ecosystem lock-in, no AI features, limited annotation tools, requires paid subscription
    \item \textbf{Market Position}: Preferred by Google Workspace users (Gmail, Docs, Drive)
    \item \textbf{ScholarFlow Differentiation}: Platform-agnostic, AI features (semantic search, summaries, chat), richer annotations
\end{itemize}

\textbf{4. Paperpal} (AI Writing Assistant, \$19.99/month)
\begin{itemize}
    \item \textbf{Strengths}: Advanced AI writing assistance, plagiarism detection, grammar/style checking, submission readiness
    \item \textbf{Weaknesses}: Expensive, focuses only on writing (not paper management), no collaboration, no citation management
    \item \textbf{Market Position}: Niche tool for manuscript preparation, not full research workflow
    \item \textbf{ScholarFlow Differentiation}: Unified platform (management + AI writing), collaboration features, lower pricing (\$9.99 Pro tier)
\end{itemize}

\textbf{5. ReadCube Papers} (Freemium, Enhanced PDF Reader)
\begin{itemize}
    \item \textbf{Strengths}: Enhanced PDF reading experience, SmartCite, article recommendations, journal integrations
    \item \textbf{Weaknesses}: No AI summaries, limited collaboration, desktop-only (poor mobile experience), outdated UI
    \item \textbf{Market Position}: Researchers who prioritize reading experience over organization
    \item \textbf{ScholarFlow Differentiation}: Mobile-first design, AI-powered multi-paper chat, modern collaboration features
\end{itemize}

\subsubsection{Adjacent Tools (Partial Overlap)}

\begin{table}[H]
    \centering
    \small
    \begin{tabular}{|l|p{5cm}|p{5cm}|}
        \hline
        \textbf{Tool}                          & \textbf{Category} & \textbf{Why Not Sufficient}  \\
        \hline
        Notion                                 &
        Note-taking, wikis, databases          &
        General-purpose tool; lacks paper-specific features (citations, PDF viewer, AI summaries) \\
        \hline
        Evernote                               &
        Note-taking, web clipper               &
        Legacy UI, no collaboration, no paper management features                                 \\
        \hline
        Google Drive                           &
        File storage, basic sharing            &
        Files only (no metadata), no AI features, poor search, no annotations                     \\
        \hline
        Overleaf                               &
        LaTeX editor, collaboration            &
        Focused on writing papers, not managing research; no AI assistance                        \\
        \hline
        Connected Papers                       &
        Citation graph visualization           &
        Single-purpose tool; no paper storage, annotations, or collaboration                      \\
        \hline
        Semantic Scholar                       &
        Paper discovery, AI summaries          &
        Discovery only (no storage/organization); limited to computer science papers              \\
        \hline
        ChatGPT                                &
        AI assistant, Q\&A                     &
        General-purpose AI; requires manual copy-paste of paper content; no persistence           \\
        \hline
        Obsidian                               &
        Markdown note-taking, knowledge graphs &
        Technical learning curve; no native PDF support, citations, or AI features                \\
        \hline
    \end{tabular}
    \caption{Adjacent Tools and Their Limitations}
    \label{table:adjacent-tools}
\end{table}

\section{Define and Desired State}

\subsection{Define (Current State)}

Based on survey data and competitive analysis, we documented the current state of research workflows:

\subsubsection{Current Workflow Fragmentation}

\begin{table}[H]
    \centering
    \small
    \begin{tabular}{|l|p{4cm}|p{6cm}|}
        \hline
        \textbf{Task}                                                      & \textbf{Tools Used} & \textbf{Pain Points} \\
        \hline
        Paper Storage                                                      &
        Desktop folders (31\%), Google Drive (31\%), Browser tabs (34.5\%) &
        No metadata, poor search, files get lost, no organization                                                       \\
        \hline
        Paper Discovery                                                    &
        Google Scholar, PubMed, arXiv                                      &
        Keyword search only (not semantic), no personalization, overwhelming results                                    \\
        \hline
        Reading \& Notes                                                   &
        PDF viewers (Adobe, browser), Notion (13.8\%)                      &
        Notes separated from papers, hard to find highlights, no AI summaries                                           \\
        \hline
        Citations                                                          &
        Zotero (20.7\%), Mendeley, manual formatting                       &
        Tedious citation formatting, export issues, no integration with writing tools                                   \\
        \hline
        Collaboration                                                      &
        Email attachments (painful), shared drives (version conflicts)     &
        No real-time editing, comment threading difficult, permission management clunky                                 \\
        \hline
    \end{tabular}
    \caption{Current State: Fragmented Research Workflow}
    \label{table:current-state}
\end{table}

\subsubsection{Quantified Pain Points}

\begin{itemize}
    \item \textbf{Tool Fragmentation}: Average researcher uses \textbf{4.2 disconnected tools} daily
    \item \textbf{Context-Switching Overhead}: \textbf{30--50\% productivity loss} due to app-switching
    \item \textbf{Search Inefficiency}: \textbf{40.6\%} struggle to find papers when needed
    \item \textbf{Organization Chaos}: \textbf{46.9\%} can't keep notes/highlights organized
    \item \textbf{Collaboration Friction}: \textbf{25\%} cite team collaboration as major pain point
    \item \textbf{Time Waste}: \textbf{40--60\%} of research time spent on administrative tasks (not actual research)
\end{itemize}

\subsection{Desired State}

ScholarFlow's vision is to consolidate fragmented workflows into a unified, AI-native research hub:

\subsubsection{Unified Research Hub}

\begin{table}[H]
    \centering
    \small
    \begin{tabular}{|l|p{10cm}|}
        \hline
        \textbf{Task}    & \textbf{ScholarFlow Solution}                                                                    \\
        \hline
        Paper Storage    &
        Single source of truth: S3-backed storage with automatic metadata extraction, organized into collections/workspaces \\
        \hline
        Paper Discovery  &
        Semantic search (pgvector embeddings), AI-powered paper recommendations, citation graph exploration (Phase 4)       \\
        \hline
        Reading \& Notes &
        In-app PDF viewer with inline annotations, AI summaries (85\% time savings), rich text editor for research notes    \\
        \hline
        Citations        &
        Auto-generated citations in 10K+ styles (APA, MLA, Chicago, IEEE, Harvard), export to LaTeX/BibTeX/RIS              \\
        \hline
        Collaboration    &
        Real-time shared workspaces with 5-tier role system, inline comment threading, email sharing with permissions       \\
        \hline
    \end{tabular}
    \caption{Desired State: Unified ScholarFlow Workflow}
    \label{table:desired-state}
\end{table}

\subsubsection{Target State Metrics}

\begin{itemize}
    \item \textbf{Single Platform}: Reduce tool count from \textbf{4.2 → 1 primary tool} (80\% reduction)
    \item \textbf{Time Savings}: Cut research admin overhead by \textbf{40--50\%} (validated by AI summary selection: 65.6\%)
    \item \textbf{Search Accuracy}: Semantic search provides \textbf{40\% accuracy improvement} over keyword search
    \item \textbf{Collaboration Efficiency}: Real-time workspaces reduce collaboration friction by \textbf{60\%} vs. email
    \item \textbf{User Satisfaction}: Target \textbf{4.5/5 satisfaction} (vs. current 3.28/5 with existing tools)
    \item \textbf{Adoption Goal}: \textbf{1,000 beta users} within 6 months, \textbf{10K users} by end of Year 1
\end{itemize}

\subsubsection{Strategic Positioning}

ScholarFlow targets the intersection of three market trends:

\begin{enumerate}
    \item \textbf{AI-First Tools}: Student researchers expect AI as default (not optional); ScholarFlow makes AI central to every workflow (search, summaries, chat)

    \item \textbf{Modern UX}: Legacy tools (Zotero/Mendeley) feel outdated; ScholarFlow provides modern UX with dark mode, responsive design, and streamlined interactions

    \item \textbf{Collaboration-Native}: Remote research is norm post-COVID; ScholarFlow builds collaboration into core (vs. bolted-on features in competitors)
\end{enumerate}

\subsection{Transition Strategy}

\begin{table}[H]
    \centering
    \small
    \begin{tabular}{|l|p{10cm}|}
        \hline
        \textbf{Phase}                                         & \textbf{Transition Actions}                                                                                                                 \\
        \hline
        \textbf{Phase 1--2} \newline \textit{(MVP Launch)}     &
        Target early adopters (UIU \textasciitilde{}60\% concentration, incl. duplicate entry); offer import from Zotero/Mendeley; emphasize AI features (semantic search, summaries) as key differentiators \\
        \hline
        \textbf{Phase 3--4} \newline \textit{(Feature Parity)} &
        Achieve feature parity with competitors (citations, browser extension, integrations); add unique features (multi-paper chat, citation graph) to create moat                                          \\
        \hline
        \textbf{Phase 5--6} \newline \textit{(Monetization)}   &
        Launch Pro tier (\$9.99/month) with advanced AI features; Enterprise tier (\$29.99/month) with SSO/team management; migrate power users from freemium                                                \\
        \hline
        \textbf{Phase 7} \newline \textit{(Enterprise)}        &
        Target institutional sales (universities, research labs); achieve SOC 2 certification; offer dedicated support and custom integrations                                                               \\
        \hline
    \end{tabular}
    \caption{Current-to-Desired State Transition Strategy}
    \label{table:transition-strategy}
\end{table}
