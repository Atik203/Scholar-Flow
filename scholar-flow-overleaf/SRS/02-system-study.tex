\chapter{System Study \& Information Gathering}

% ============================================
% 2.1 SURVEY METHODOLOGY
% ============================================
\section{Survey Methodology}

\subsection{Survey Design and Objectives}

To validate the market need and gather comprehensive user requirements, a structured survey was conducted using Google Forms. The survey aimed to:

\begin{enumerate}
    \item \textbf{Understand Current Research Workflows}: Identify tools currently used and satisfaction levels
    \item \textbf{Identify Pain Points}: Discover challenges researchers face in paper management
    \item \textbf{Validate Feature Priorities}: Assess demand for proposed ScholarFlow features
    \item \textbf{Measure Market Interest}: Gauge adoption likelihood and willingness to pay
    \item \textbf{Inform Product Strategy}: Guide feature prioritization and MVP scope
\end{enumerate}

\subsection{Survey Structure}

The survey consisted of 21 comprehensive questions organized into four main categories:

\begin{table}[H]
    \centering
    \begin{tabularx}{\textwidth}{|l|c|X|}
        \hline
        \textbf{Question Category}  & \textbf{Count} & \textbf{Purpose}                                                 \\
        \hline
        \textbf{Demographics}       & 5              &
        Understand respondent roles, academic levels, fields of study, age groups, and institutional affiliations       \\
        \hline
        \textbf{Current Tool Usage} & 6              &
        Assess existing tools, satisfaction levels, pain points, and collaboration patterns                             \\
        \hline
        \textbf{Feature Priorities} & 10             &
        Measure interest in AI features, collaboration tools, analytics, citation management, and advanced capabilities \\
        \hline
        \textbf{Adoption Intent}    & 6              &
        Gauge likelihood of adoption, willingness to pay, concerns, and feedback mechanisms                             \\
        \hline
    \end{tabularx}
    \caption{Survey Question Categories}
    \label{table:survey-categories}
\end{table}

\subsection{Data Collection}

\begin{itemize}[leftmargin=*]
    \item \textbf{Survey Period}: November - December 2025
    \item \textbf{Total Responses}: 32 survey respondents
    \item \textbf{Distribution Method}: Online via email, university forums, and academic social networks
    \item \textbf{Target Population}: Students, faculty, and researchers across multiple universities
    \item \textbf{Geographic Focus}: Primarily Bangladesh, with concentration at United International University (UIU)
    \item \textbf{Response Rate}: Sufficient for preliminary feasibility analysis and MVP validation
\end{itemize}

% ============================================
% 2.2 DEMOGRAPHIC ANALYSIS
% ============================================
\section{Demographic Analysis}

\subsection{Role Distribution}

The survey captured a diverse range of academic roles, with a strong concentration in the target undergraduate segment:

\begin{figure}[H]
    \centering
    \includegraphics[width=0.8\textwidth]{../response_image/1.png}
    \caption{Survey Respondents by Role (n=32)}
    \label{fig:role-distribution}
\end{figure}

\textbf{Key Findings:}
\begin{itemize}
    \item \textbf{78.1\% Undergraduate Students}: Dominant user segment, validating focus on student-centric features
    \item \textbf{12.5\% Faculty/Professors}: Potential institutional buyers and influencers
    \item \textbf{\textasciitilde{}3.1\% Postgraduate (Masters/MPhil)}: Emerging segment requiring advanced collaboration features
    \item \textbf{\textasciitilde{}3.1\% PhD Researchers}: Power users needing extensive paper libraries and citation tools
    \item \textbf{\textasciitilde{}3.1\% Other}: Research assistants and other academic roles
\end{itemize}

\textbf{Strategic Implication}: Product messaging and feature prioritization should focus heavily on undergraduate research workflows while ensuring scalability for graduate-level requirements.

\subsection{Field of Study Distribution}

\begin{figure}[H]
    \centering
    \includegraphics[width=0.8\textwidth]{../response_image/2.png}
    \caption{Respondents by Academic Field (n=32)}
    \label{fig:field-distribution}
\end{figure}

\textbf{Key Findings:}
\begin{itemize}
    \item \textbf{34.4\% Computer Science}: Highest field concentration
    \item \textbf{\textasciitilde{}18\% Other CS-related fields}: Additional technology-centric audience
    \item \textbf{6.3\% Electrical Engineering} and \textbf{6.3\% Data Science}: Strong technical alignment
    \item Remaining fields (Civil, Pharmaceutical Science, and others): \textbf{\textasciitilde{}3.1\% each}
\end{itemize}

\textbf{Strategic Implication}: CS/IT concentration suggests low technical friction for adoption. Marketing should emphasize modern UI/UX and AI features that resonate with tech-forward users.

\subsection{Academic Level Distribution}

\begin{figure}[H]
    \centering
    \includegraphics[width=0.8\textwidth]{../response_image/3.png}
    \caption{Respondents by Academic Year (n=32)}
    \label{fig:academic-level}
\end{figure}

\textbf{Key Findings:}
\begin{itemize}
    \item \textbf{53.1\% 3rd-year undergraduates}: Peak research intensity period (capstone projects, thesis preparation)
    \item \textbf{15.6\% 2nd-year undergraduates}: Beginning research exposure
    \item \textbf{\textasciitilde{}3.1\% 1st-year undergraduates}: Early-stage research exposure
    \item \textbf{12.5\% Not a student}: Faculty/other roles included in the sample
    \item \textbf{\textasciitilde{}6.3\% Masters/MPhil} and \textbf{\textasciitilde{}6.3\% PhD}: Advanced research segments
\end{itemize}

\textbf{Strategic Implication}: Campus rollout should target 3rd-year students first, then expand to other years through peer referrals.

\subsection{Age Distribution}

\begin{figure}[H]
    \centering
    \includegraphics[width=0.8\textwidth]{../response_image/4.png}
    \caption{Respondents by Age Group (n=32)}
    \label{fig:age-distribution}
\end{figure}

\textbf{Key Findings:}
\begin{itemize}
    \item \textbf{68.8\% Ages 22--25}: Gen-Z digital natives expecting modern SaaS experiences
    \item \textbf{12.5\% Ages 18--21}: Younger undergraduates beginning research journeys
    \item \textbf{12.5\% Ages 26--30}: Graduate students and professionals
    \item \textbf{\textasciitilde{}3.1\% Ages 31--35} and \textbf{\textasciitilde{}3.1\% Above 40}: Faculty/older professionals
\end{itemize}

\textbf{Strategic Implication}: Design must prioritize mobile-first responsive layouts, dark mode, and social collaboration features aligned with Gen-Z preferences.

\subsection{Institutional Distribution}

\begin{figure}[H]
    \centering
    \includegraphics[width=0.8\textwidth]{../response_image/5.png}
    \caption{Respondents by University (n=32)}
    \label{fig:university-distribution}
\end{figure}

\textbf{Key Findings:}
\begin{itemize}
    \item \textbf{\textasciitilde{}60\% United International University (UIU)}: Strong concentration enables focused beta testing (includes duplicate UIU entry)
    \item \textbf{10\% North South University (NSU)}: Secondary concentration
    \item \textbf{\textasciitilde{}30\% Other Universities}: AIUB, IUB, Khulna University, and others
\end{itemize}

extbf{Strategic Implication}: Phase 1 launch should focus on UIU as the pilot campus, leveraging high response concentration for rapid word-of-mouth growth while expanding outreach for broader validation.

% ============================================
% 2.3 CURRENT TOOL LANDSCAPE
% ============================================
\section{Current Tool Landscape}

\subsection{Existing Tool Usage Patterns}

\begin{figure}[H]
    \centering
    \includegraphics[width=0.8\textwidth]{../response_image/6.png}
    \caption{Current Research Management Tools Used (n=32, multiple selections allowed)}
    \label{fig:current-tools}
\end{figure}

\textbf{Key Findings:}
\begin{itemize}
    \item \textbf{34.4\% Browser PDFs}: Basic reading without organization or collaboration
    \item \textbf{31.3\% Google Drive / OneDrive}: Storage-first workflows lacking research context
    \item \textbf{31.3\% Local folders}: Fragmented organization without metadata/search
    \item \textbf{28.1\% No specific tool}: Ad-hoc workflows and low tool lock-in
    \item \textbf{15.6\% Zotero} and \textbf{15.6\% built-in reference managers}: Citation-focused tools with limited modern collaboration
    \item \textbf{9.4\% EndNote} and \textbf{9.4\% Paperpile}; \textbf{6.3\% Mendeley}: Smaller legacy segments
\end{itemize}

extbf{Critical Insight}: High reliance on ad-hoc and storage-only workflows indicates low switching costs and strong opportunity for a freemium onboarding strategy.

\subsection{Current Satisfaction Levels}

\begin{figure}[H]
    \centering
    \includegraphics[width=0.8\textwidth]{../response_image/7.png}
    \caption{Satisfaction with Current Research Tools (n=32)}
    \label{fig:satisfaction-levels}
\end{figure}

extbf{Average Satisfaction Score: 3.28/5.00}

\textbf{Distribution:}
\begin{itemize}
    \item \textbf{40.6\% Neutral (3/5)}: Indifferent or using multiple fragmented tools
    \item \textbf{31.3\% Satisfied (4/5)}: Functional but see room for improvement
    \item \textbf{15.6\% Rating 2/5} and \textbf{3.1\% Rating 1/5}: Actively seeking better alternatives
    \item \textbf{9.4\% Very satisfied (5/5)}: Power users who have optimized workflows
\end{itemize}

extbf{Strategic Implication}: 18.7\% dissatisfaction (1--2/5) + 40.6\% neutrality (3/5) = \textbf{59.3\% addressable market} open to switching to a superior solution.

\subsection{Primary Pain Points}

\begin{figure}[H]
    \centering
    \includegraphics[width=0.8\textwidth]{../response_image/8.png}
    \caption{Biggest Challenges in Research Paper Management (n=32)}
    \label{fig:pain-points}
\end{figure}

\textbf{Top 5 Challenges:}
\begin{enumerate}
    \item \textbf{46.9\% Difficulty taking and keeping notes}: Notes are scattered and hard to maintain
    \item \textbf{40.6\% Hard to find papers later}: Poor discoverability across tools and folders
    \item \textbf{34.4\% Hard to keep papers organized}: Lack of structured collections/tags
    \item \textbf{28.1\% Syncing across devices is inconsistent}: Workflow breaks across laptop/mobile
    \item \textbf{25\% Collaboration challenges}: Sharing and discussion are fragmented
\end{enumerate}

extbf{Critical Insight}: The top pain points cluster around note-taking, discoverability, and organization, directly validating ScholarFlow's core value proposition.

\subsection{Collaboration Frequency}

\begin{figure}[H]
    \centering
    \includegraphics[width=0.8\textwidth]{../response_image/9.png}
    \caption{Collaboration Practices for Research Papers (n=32)}
    \label{fig:collaboration-frequency}
\end{figure}

\textbf{Key Findings:}
\begin{itemize}
    \item \textbf{50\% Mostly work alone}: Personal organization is the first value driver
    \item \textbf{25\% Share PDFs via email/messaging apps}: Collaboration is informal and unstructured
    \item \textbf{9.4\% Use shared cloud folders} and \textbf{9.4\% use dedicated collaboration tools}: Small but meaningful collaboration segment
    \item \textbf{\textasciitilde{}6.3\% Comment directly in shared PDFs}: Lightweight discussion behavior exists
\end{itemize}

\textbf{Strategic Implication}: Product should support "from solo to shared" workflows, allowing individual users to easily invite collaborators when needed.

% ============================================
% 2.4 MARKET VALIDATION
% ============================================
\section{Market Validation}

\subsection{Perceived Need for Dedicated Solution}

\begin{figure}[H]
    \centering
    \includegraphics[width=0.8\textwidth]{../response_image/10.png}
    \caption{Need for Dedicated Research Paper Management Platform (n=32)}
    \label{fig:market-need}
\end{figure}

\textbf{Key Findings:}
\begin{itemize}
    \item \textbf{75\% At least a moderate need} (Moderate + Strong + Very Strong)
          \begin{itemize}
              \item 28.1\% Moderate need
              \item 28.1\% Strong need
              \item 18.8\% Very strong need
          \end{itemize}
    \item \textbf{25\% Slight-to-No need} (Slight need + No need)
\end{itemize}

\textbf{Critical Validation}: Nearly \textbf{3 out of 4 researchers} express moderate-to-extreme need for a dedicated platform, confirming significant market opportunity.

\subsection{Interest in ScholarFlow Specifically}

\begin{figure}[H]
    \centering
    \includegraphics[width=0.8\textwidth]{../response_image/11.png}
    \caption{Interest in Using ScholarFlow Platform (n=32)}
    \label{fig:product-interest}
\end{figure}

\textbf{Key Findings:}
\begin{itemize}
    \item \textbf{81.3\% Moderate-to-High Interest} (Moderately + Very + Extremely Interested)
          \begin{itemize}
              \item 18.8\% Moderately interested
              \item 53.1\% Very interested
              \item 9.4\% Extremely interested
          \end{itemize}
    \item \textbf{18.8\% Low interest} (Slightly interested + Not interested)
\end{itemize}

extbf{Critical Validation}: A strong majority of respondents (81.3\%) express moderate-to-high interest, indicating high conversion potential from awareness to adoption.

\subsection{Adoption Likelihood (Free Tier)}

\begin{figure}[H]
    \centering
    \includegraphics[width=0.8\textwidth]{../response_image/12.png}
    \caption{Likelihood of Trying ScholarFlow Free Tier (n=32)}
    \label{fig:adoption-likelihood}
\end{figure}

\textbf{Key Findings:}
\begin{itemize}
    \item \textbf{62.6\% Likely-to-Very Likely} to try free tier
          \begin{itemize}
              \item 31.3\% Very likely
              \item 31.3\% Likely
          \end{itemize}
    \item \textbf{21.9\% Not sure} (fence-sitters requiring more marketing)
    \item \textbf{15.7\% Unlikely} (Unlikely + Very unlikely)
\end{itemize}

\textbf{Strategic Implication}: Generous free tier is critical for reducing friction and driving viral adoption through student networks.

\subsection{Summary: Market Validation Metrics}

\begin{table}[H]
    \centering
    \begin{tabularx}{\textwidth}{|l|c|X|}
        \hline
        \textbf{Metric}                 & \textbf{Value}        & \textbf{Interpretation} \\
        \hline
        extbf{Perceived Need}           & 75\%                  &
        Strong market demand for dedicated solution                                       \\
        \hline
        extbf{Product Interest}         & 81.3\%                &
        High immediate conversion potential                                               \\
        \hline
        extbf{Trial Likelihood}         & 62.6\%                &
        Freemium model will drive adoption                                                \\
        \hline
        extbf{Current Satisfaction}     & 3.28/5                &
        Significant room for disruption                                                   \\
        \hline
        extbf{"Not Sure" Segment}       & 21.9\%                &
        Opportunity to convert fence-sitters through marketing                            \\
        \hline
        extbf{Target Demographic}       & 78.1\%                &
        Undergraduates = clear product-market fit                                         \\
        \hline
        extbf{Geographic Concentration} & \textasciitilde{}60\% &
        UIU pilot enables rapid iteration and feedback                                    \\
        \hline
    \end{tabularx}
    \caption{Market Validation Summary Metrics}
    \label{table:validation-metrics}
\end{table}

\textbf{Conclusion}: Survey data provides \textit{strong validation} for ScholarFlow's market opportunity, with clear demand signals, identified pain points, and high adoption intent among the target undergraduate segment.
